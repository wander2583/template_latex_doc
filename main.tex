\documentclass[a4paper,12pt]{article}

% --- Pacotes Fundamentais ---
\usepackage[utf8]{inputenc}      % Codificação de caracteres
\usepackage[T1]{fontenc}         % Codificação de fontes
\usepackage[brazil]{babel}       % Idioma português
\usepackage{geometry}            % Margens
\usepackage{fancyhdr}            % Para Cabeçalhos e Rodapés personalizados
\usepackage{graphicx}            % Para inserir o logo
\usepackage{setspace}            % Espaçamento entre linhas
\usepackage{lipsum}              % Para texto de exemplo (pode remover depois)
\usepackage{times}               % Fonte Times New Roman (clássica para academias)

% --- Configuração das Margens ---
\geometry{
    a4paper,
    top=3cm,
    bottom=2.5cm,
    left=3cm,
    right=2.5cm,
    headheight=1.5cm % Espaço para o cabeçalho
}

% -Presidente-
\pagestyle{fancy}
\fancyhf{} % Limpa cabeçalhos e rodapés padrão

% CABEÇALHO
\fancyhead[L]{%
    % Se tiver um logo, descomente a linha abaixo e envie o arquivo imagem
    % \includegraphics[height=1.5cm]{logo_abarcle.png} 
    \textbf{ABARCLE} % Texto alternativo caso não tenha logo agora
}
\fancyhead[C]{%
    \textbf{\large ACADEMIA BARCARENENSE DE LETRAS}\\
    \small \textit{Cultivando a cultura e a história no coração do Pará}\\
    \footnotesize Fundada em [Ano de Fundação]
}
\fancyhead[R]{%
    \small Barcarena, PA
}

% Linha horizontal abaixo do cabeçalho
\renewcommand{\headrulewidth}{1pt}

% RODAPÉ
\fancyfoot[C]{%
    \footnotesize
    Av. Cronge da Silveira, s/n - Barcarena/PA - Brasil\\
    Contato: contato@abarcle.org.br | www.abarcle.org.br\\
    \textit{Página \thepage}
}

% Linha horizontal acima do rodapé
\renewcommand{\footrulewidth}{0.5pt}

% --- Início do Documento ---
\begin{document}

% Título do Texto Específico
\begin{center}
	\vspace*{0.5cm}
	{\Large \textbf{Ecos do Rio Pará: A Literatura na Amazônia}} \\
	\vspace{0.3cm}
	\textit{Por [Nome do Acadêmico/Autor]}
\end{center}
\vspace{1cm}

% Corpo do Texto
\onehalfspacing % Espaçamento 1.5

Prezados Confrades e Confreiras,

É com grande estima que registramos neste documento a importância da preservação da memória barcarenense. A literatura, em nossa cidade, flui como as águas do Rio Pará: ora calma e reflexiva, ora revolta e cheia de transformações, como nos tempos da Cabanagem.

A Academia Barcarenense de Letras (ABARCLE) tem o dever cívico e cultural de fomentar a escrita e a leitura, servindo como um farol para as novas gerações. Nossas lendas, nossa culinária e nossa gente são a tinta com a qual escrevemos a história deste município vibrante.

Este documento serve como modelo oficial para as comunicações, crônicas e editais de nossa instituição, garantindo a sobriedade e a padronização necessárias para os nossos arquivos históricos.

% Texto de preenchimento (apenas para ocupar espaço no exemplo)
\vspace{0.5cm}
\lipsum[1-2]

\vspace{1cm}
\begin{flushright}
	Barcarena, \today.

	\vspace{1.5cm}
	\textbf{um inutiu qualquer da ABARCLE}\\
	Academia Barcarenense de Letras
\end{flushright}

\end{document}
