\documentclass[a4paper,12pt]{article}
\usepackage[utf8]{inputenc}
\usepackage[T1]{fontenc}
\usepackage[brazil]{babel}
\usepackage{graphicx}
\usepackage{geometry}
\usepackage{fancyhdr}
\usepackage{xcolor}
\usepackage{times}
\usepackage{setspace}
\usepackage{hyperref}

% --- Configuração Visual ---
\renewcommand{\familydefault}{\sfdefault}
\geometry{a4paper, top=3cm, bottom=2cm, left=2cm, right=2cm, headheight=4.5cm}
\definecolor{barcarenaOrange}{RGB}{210, 105, 30}
\definecolor{laranjaTitulo}{HTML}{E77014}

% --- Header/Footer ---
\pagestyle{fancy}
\fancyhf{}
\renewcommand{\headrulewidth}{0pt}
\renewcommand{\footrulewidth}{0pt}

\fancyhead[C]{%
    \includegraphics[width=2.5cm]{./img/logo.jpg} \\ 
    \vspace{0.2cm}
    \textbf{\large ACADEMIA BARCARENENSE DE LETRAS} \\
    LEI MUNICIPAL Nº 2260/2021 \\
    BARCARENA -- PA \\ 
    \noindent\textcolor{laranjaTitulo}{\rule{\textwidth}{3pt}}
}

\fancyfoot[C]{%
    \noindent\textcolor{laranjaTitulo}{\rule{\textwidth}{3pt}} \\
    \small
    \textcolor{black}{Travessa 7 DE SETEMBRO, Bairro Comercial - BARCARENA-PA - CEP 667400-12} \\
    \href{https://instagram.com/abarcle_oficial}{\texttt{https://instagram.com/abarcle\_oficial}} \\
    \vspace{0.1cm}
    % \textcolor{barcarenaOrange}{\rule{\textwidth}{3pt}}
}

\title{Sonde-me por favor}
\author{Sebastiao J. Cardoso}

\begin{document}

% \maketitle

\begin{onehalfspace}
	Esprema meus neurônios, destrave a recepção, libere a transmissão.
Sei lá...
Ajude-me a ordenar saltos elétrico-químicos, receita-me um insemino,
De puro alento...
Suplico-te! Receite-me um sustento sináptico urgente para que eu
Neuroplasticide um bocado de aprendizagens e adaptações outras,
Capazes de me impulsionar um dia dopaminérgico.
Sabes? Destes de puro construir e fruir.
Não! Antes receite-me um sistema límbico inteiro.
Use para isso Axônio envoltos numa nuvem poderosa de Mielina,
Assim, diz o cosmos, haverá uma transmissão mais veloz do impulso nervoso.
Não, não estou em mania ou na prima dela, sei que não.
Bem... Sei lá aonde vou,
Mas suponho que ando por sobre o córtex frontal e viso a homeostase
Por igual! % Placeholder do Conteúdo
\end{onehalfspace}

\end{document}
